\documentclass[document]{xmgr}
% Jeśli nowe rozdziały mają się zaczynać na stronach
% nieparzystych:
%\documentclass[openright]{xmgr}

%\defaultfontfeatures{Scale=MatchLowercase}
%\setmainfont[Numbers=OldStyle,Ligatures=TeX]{Minion Pro}
%\setsansfont[Numbers=OldStyle,Ligatures=TeX]{Myriad Pro}
% for fontspec version < 2.0
\setmainfont[Numbers=OldStyle,Mapping=tex-text]{Arial}
\setsansfont[Numbers=OldStyle,Mapping=tex-text]{Times New Roman}
\usepackage{amsmath}
\usepackage{amsfonts}
\usepackage{listings} 
\usepackage{color}
\usepackage{graphicx}

\setmonofont[Scale=0.75]{Monaco}

% Opcjonalnie identyfikator dokumentu
% drukowany tylko z włączoną opcją 'brudnopis':
\wersja   {wersja wstępna [\ymdtoday]}

\author   {Szymon Rękawek}
\nralbumu {206288}
\email    {rekawekszymon@gmail.com}

\title    {Gra Thuego}
\date     {01.01.2016}
\miejsce  {Gdańsk}

\opiekun  {prof. dr hab. T. Dzido}

% dodatkowe polecenia
%\renewcommand{\filename}[1]{\texttt{#1}}
\definecolor{stress}{cmyk}{0,1,0.13,0} % RubineRed
\definecolor{topic}{cmyk}{0.98,0.13,0,0.43} % MidnightBlue

\begin{document}

% streszczenie
\begin{abstract}

\end{abstract}

% słowa kluczowe
\keywords{Thue}

% tytuł i spis treści
\maketitle

% wstęp
\introduction

Tematem niniejszej pracy jest Gra Thuego. Axel Thue był norweskim matematykiem żyjącym w latach 1863 - 1922, znanym z prac z zakresu kombinatoryki.

Thue pracował nad problemami, powstałymi w wyniku badań nad sekwencjami symboli. Praca Thue [1] opisywała problem, który autor nazwał \textit{nieredukowalne słowa} (\textit{irreducible words}). Poświęca w niej szczególną uwagę dwu i trzy literowym przypadkom. W skrócie wprowadza pojęcie znane obecnie jako \textit{Thue-Morse word} i pokazuje, że nieskończone słowa bez nasunięć (\textit{Overlap-free}) są pochodnymi tej sekwencji. W swoich pracach definiuje kolejną strukturę, a mianowicie infinite \textit{Square-Free word}, oraz przedstawia sposoby generowania nieskończenie długich słów wolnych zarówno od kwadratów jak i nasunięć.

Gra która powstała na podstawie teorii Thuego w skrócie polegała będzie na utworzeniu jak najdłuższego ciągu znaków nad określonym z góry alfabetem. Zależnie od trybu gry, kończyć się ona będzie w momencie gdy pojawi się zdefiniowany na początku rodzaj powtórzenia w tworzonym przez nas, bądź algorytm ciągu. Jednym z trybów gry jest walka komputera przeciwko niemu samemu, po takiej rozgrywce przedstawione zostaną złożoności czasowe oraz wnioski wynikające z obranej przez oponentów taktyki. 

Ostatnia część pracy poświęcona jest analizie algorytmów zarówno pod względem czasu ich wykonywania jak i zdolności do przewidywania ruchów przeciwnika.

\chapter{Teorie Axela Thue na temat sekwencji symboli}
Zanim przejdziemy do twierdzeń, koniecznym jest wyjaśnienie podstawowych twierdzeń, które w dalszej części pracy będą wielokrotnie wykorzystywane. Niżej wyjaśnione zostały sposoby generowania nieskończenie długich ciągów składających się z dwóch symboli, nie zawierających nasunięć. Po czym ukazane zostały metody tworzenia nieskończenie długich ciągów na trzech znakach, wolnych od kwadratów. Na koniec części teoretycznej przedstawiony został pomysł gry dla dwóch graczy wykorzystujący powyższe własności.


\section{Definicje}
\begin{itemize}
\item Alfabet jest skończonym zbiorem symboli lub liter.
\item Słowo alfabetu $A$ jest skończoną sekwencją elementów z $A$. 
\item Długość słowa $\omega$ jest reprezentowana przez $|\omega|$.
\item Puste słowo o długości $0$ jest reprezentowane przez $\varepsilon$.
\item Czynnik (factor) słowa $\omega$ jest słowem $u$, które występuje wewnątrz $\omega$ formie $\omega = xuy$, podczas gdy $x$ oraz $y$ również są słowami tego alfabetu.
\item Kwadrat (square) jest niepustym słowem w formie $uu$, gdzie $u$ jest niepuste.
\item Square-free, słowo jest wolne od kwadratów, jeśli żaden z jego czynników nie jest kwadratem.
\item Nasunięcie (Overlap) jest słowem w formie $xuxux$, gdzie $x$ jest niepuste. Nazwa pojęcia wzięła się z tego, że $xux$ występuje dwa razy w $xuxux$. Pierwszy raz jako prefiks (początkowy czynnik) oraz jako sufiks (końcowy czynnik) i te dwa wystąpienia mają wspólną część - centralne $x$, a więc \textit{nasuwają} się na siebie.
\item Overlap-free - słowo w którym żaden z czynników nie nasuwa się na siebie.
W definicji Axela Thue słowo $\omega$ w alfabecie długości n jest nieredukowalne jeśli jakiekolwiek dwa wystąpienia tego samego słowa jako czynnik wewnątrz $\omega$ są zawsze oddzielone od siebie przez $n-2$ liter. Oznacza to, że nieredukowalne dwuliterowe słowo jest bez nasunięć i nieredukowalne trzyliterowe słowo jest bez kwadratów.
\item Morfizm - mapowanie obiektu z jednej matematycznej struktury w inną.
\end{itemize}



\section{Thue-Morse word}
Rozważmy nieskończone słowo

{\centering $01101001100101101001011001101001 ...$ \par}

Zostało ono nazwane po Thue, który badał jego właściwości w referacie z 1906 roku, oraz Morsie, który odkrył je na nowo w latach 20 XIX wieku. Słowo Thue-Morse'a występuje również o wiele wcześniej w wiadomościach Prouheta[200] z Francuską Akademią Nauk w 1851 roku. Rzeczywiście Prouhet podał więcej ogólnych konstrukcji, uzyskując nie tylko słowo Thue-Morse'a, ale całą rodzinę słów na większych alfabetach mających inne interesujące właściwości. Słowa te czasami odnoszą się do ogólnych słow Thue-Morse'a lub słów Prouheta.

Niech $A = \{a, b\}$ będzie dwuliterowym alfabetem. Rozważmy morfizm $\mu$ z monoidu $A*$, który definiuje się następująco:



\begin{tabbing}

\hspace{8em}\= $\mu(a) = ab$,\hspace{7em}\= $\mu(b) = ba$\\
Dla $n \geq 0$:\\
\> $u_n = \mu^n(a)$,\> $v_n = \mu^n(b)$\\
Wtedy:\\
\> $u_0 = a$ \> $v_0 = b$\\
\> $u_1 = ab$ \> $v_1 = ba$\\
\> $u_2 = abba$	 \> $v_2 = baab$\\
\> $u_3 = abbabaab$ \> $v_3 = baababba$\\
Wzór ogólny:\\
\> $u_{n+1} = u_n v_n,$ \> $v_{n+1} = v_n u_n$\\
oraz:\\
\> $u_n = \overline{v}_n,$ \> $v_n = \overline{u}_n$
\end{tabbing}

gdzie $\overline{\omega}$ jest uzyskiwane z $\omega$ przez zamianę $a$ oraz $b$. Słowa $u_n$ i $v_n$ są często nazywane \textit{Blokami morsa}. Można łatwo zauważyc że $u_{2n}$ oraz $v_{2n}$ są palindromami oraz to że $u_{2n+1} = \neg v_{2n+1}$, gdzie $\neg w$ jest negacją $w$. Morfizm $\mu$ może być rozszerzony do nieskończonych słow, które mają dwa stałe punkty:


{\centering 
$t = abbabaabbaababbabaab... = \mu(t)$ \\
$~t = baababbaabbabaababba... = \mu(~t)$ 
\par}



Przedstawione powyżej słowo $t$ jest sekwencją Thue-Morse'a. Jest wiele innych sposobów na stworzenie tego słowa. Niech $t_n$ będzie n-tym symbolem w $t$, zaczynając od $n = 0$. Wtedy można pokazać, że:

\[
t_n=
\left \{
\begin{tabular}{ccc}
$a\ if\ d_1(n) \equiv (mod\ 2)$ \\
$a\ if\ d_1(n) \equiv (mod\ 2)$
\end{tabular}
\right \}
\]

\begin{align*}
  &\phantom{{}\leq{}} \text{gdzie } d_1(n) \text{ jest liczbą bitów równych 1 w binarnej reprezentacji } n \text{.}  \\
  & \text{Dla } n \leq 12 \text{ oraz } n \in \mathbb{N} \text{ generowane jest następujące słowo:}  \\
\end{align*}

\begin{tabbing}
$bin(0)$\hspace{1em} \= $= 0,$\hspace{7em} \= $d_1 (0)$\hspace{1em} \= $= 0\ mod\ 2 = 0 \to a$ \\
$bin(1)$ \> $= 1,$ \> $d_1 (1)$ \> $= 1\ mod\ 2 = 1 \to b$\\
$bin(2)$ \> $= 10,$ \> $d_1 (2)$ \> $= 1\ mod\ 2 = 1 \to b$\\
$bin(3)$ \> $= 11,$ \> $d_1 (3)$ \> $= 2\ mod\ 2 = 0 \to a$\\
$bin(4)$ \> $= 100,$ \> $d_1 (4)$ \> $= 1\ mod\ 2 = 1 \to b$\\
$bin(5)$ \> $= 101,$ \> $d_1 (5)$ \> $= 2\ mod\ 2 = 0 \to a$\\
$bin(6)$ \> $= 110,$ \> $d_1 (6)$ \> $= 2\ mod\ 2 = 0 \to a$\\
$bin(7)$ \> $= 111,$ \> $d_1 (7)$ \> $= 3\ mod\ 2 = 1 \to b$\\
$bin(8)$ \> $= 1000,$ \> $d_1 (8)$ \> $= 1\ mod\ 2 = 1 \to b$\\
$bin(9)$ \> $= 1001,$ \> $d_1 (9)$ \> $= 2\ mod\ 2 = 0 \to a$\\
$bin(10)$ \> $= 1010,$	\> $d_1 (10)$ \> $= 2\ mod\ 2 = 0 \to a$\\
$bin(11)$ \> $= 1011,$ 	\> $d_1 (11)$ \> $= 3\ mod\ 2 = 1 \to b$\\
$bin(12)$ \> $= 1100,$ 	\> $d_1 (12)$ \> $= 2\ mod\ 2 = 0 \to a$
\end{tabbing}

{\centering $t = abbabaabbaaba$ \par}

W konsekwencji istnieje skończony automat obliczający wartości $t_n$. Automat ten ma dwa stany końcowe $0$ oraz $1$. Na początku czyta łańcuch znaków $bin(n)$ od lewej do prawej, zaczynając od $n = 0$. Ostateczny stan równy jest $0$ lub $1$ i definiuje czy $t_n$ jest równe $a$ lub $b$. W skrócie obliczenie jakie wykonuje automat to $d_1(n)\ modulo\ 2$.


\section{Square-free word}
Łatwo można zauważyc, że jedynymi słowami bez kwadratów w alfabecie $A = \{a, b\}$ są: $a, b, ab, ba, aba, bab$. Istnieje jednak dowolnie długi ciąg znaków wolny od kwadratów dla słów nad alfabetem trzyliterowym. By stworzyć dowolne słowo wolne od kwadratów Thue wymyślił następujący algorytm.\\
Mając alfabet $A = \{a, b, c\}$ należy zastąpić każde wystąpienie litery $a$ przez $abac$,  $b$ przez $babc$ oraz $c$ przez $bcac$, jeśli jest poprzedzone przez $a$ lub $acbc$, jeśli jest poprzedzone przez $b$. Zaczynając od litery $a$ otrzymujemy nieskończone słowo które nie zawiera kwadratów.

{\centering $abacbabcabacbcacbabcabacbabcacbcabacbabc...$ \par}

W 1912 roku Axel Thue wymyślił inny sposób na generowanie nieskończonego słowa bez kwadratów na trzech literach z użyciem następującego morfizmu. 
\begin{itemize}
\item $a \to abcab$
\item $b \to acabcb$
\item $c \to acbcacb$
\end{itemize}


Po raz kolejny zastępujemy każde wystąpienie z naszych liter przez zdefiniowane sekwencje. Jest to dość skomplikowana struktura, zsumowana długość łańcuchów wynosi 18. A Carpi [7] dowiódł, że morfizm na alfabecie składającym się z trzech liter tworzący słowa wolne od kwadratów musi mieć długość równą co najmniej $18$.

% - Bob wybiera indeks w ciągu, natomiast Alicja wybiera symbol z wcześniej ustalonego zbioru $A$.
\section{Thue online}
Praca J. Grytczuka, P. Szafrugi i M. Zmarza pod tytułem Online version of theorem of Thue[8] opisuje wersję online teorii Thuego. Jest to gra dla dwóch graczy Boba oraz Alicji. Podczas rozgrywki Alicja i Bob naprzemiennie wykonując swoje ruchy tworzą ciąg bez kwadratów. Celem Boba jest jak najszybsze skończenie rozgrywki poprzez utworzenie kwadratu, Alicja natomiast musi tego unikać.

Wartym wspomnienia jest uproszczony tryb gry, podczas którego mamy dwóch graczy - Alicję i Boba, tak jak zostało wspomniane, tylko Alicji zależy na tym by uniknąć kwadratów. Rozgrywka polega na tym, że Alicja i Bob wybierają na przemian symbole z ustalonego zbioru $A$, oraz dopisują je na końcu istniejącego ciągu. W momencie, gdy pojawia się kwadrat $aa$, jego druga część, czyli w naszym przypadku prawe $a$, zostaje usunięte. Zostało udowodnione w [9] J. Grytczuk, J. Kozik, P. Micek, New approach to nonrepetitive sequences. Random Structures Algorithms, DOI 10.1002/rsa.20411. , że Alicja jest wstanie stworzyć dowolnie długi ciąg bez kwadratów, nie zważając na ruchy Boba. Powyższe jest jednak możliwe pod warunkiem, że moc zbioru $A$ wynosi conajmniej $8$.

Innym typem gry przedstawionym w pracy[8] jest Online Thue game. Po raz  
kolejny gracze wykonują swoje ruchy na przemian. W swojej rundzie Bob wybiera indeks w istniejącym ciągu $S$, który jest sprecyzowany przez 
liczbę $i \in \{0, 1, ..., n\}$, następnie Alicja wybiera symbol $x \in A
$, który jest wstawiany jedną pozycję w prawo od $s_i$, dając nam nową sekwencję $S' = s_1, ..., s_i, x, s_{i+1}, ...,s_n$ w momencie gdy $i = 0$, $x$ jest ustawiany na początku $S$. Celem Boba jest zmuszenie Alicji do stworzenia kwadratu, podczas gdy Alicja unika tego najdłużej jak to możliwe. Na przykład, jeśli ustalimy, że $A = \{a, b, c\}$ i $S = acbc$, wtedy Bob wybierając indeks $i = 1$ nie daje Alicji możliwości wybrania symbolu, który nie stworzyłby kwadratu. Rzeczywiście wybierając jakikolwiek $x \in A$ doprowadza do utworzenia kwadratu w $S'$: $\textbf{aa}cb, a\textbf{bcbc}, a\textbf{cc}bc$. W przypadku gdy Bob obierze dobrą strategię, rozgrywka na 3 symbolach skończy się na ciągu o długości $\leq 5$, nie ważne jak dobrą strategię obierze Alicja. Oczywiście im większą moc ma zbiór $A$, tym więcej ruchów, będzie potrzebował Bob, by zakończyć rozgrywkę. 

[9] A. Ku ̈ndgen and M. J. Pelsmajer, Nonrepetitive colorings of graphs of bounded treewidth, Discrete Math. 308 (2008), 4473–4478. [10] J. Bara ́t, P. P. Varju ́. On square-free vertex colorings of graphs. Studia Sci. Math. Hungar. 44 (2007) 411–422. 

Grytczuk, Szafruga i Zmarza[8] wysnuli teorię, że istnieje strategia dla Alicji gwarantująca jej utworzenie dowolnie długiej gry w online Thue game na zbiorze o 12 symboli. Teorię swoją oparli o prace Ku ̈ndegena i Pelsmajera[9], oraz Baráta and Varju[10], na temat niepowtarzalnych kolorowań grafów planarnych (outerplanar graphs?).

Autorzy [8] nie zamkneli do końca problemu i zauważyli, że może istnieć strategia dla Alicji, która pozwoliłaby jej na dowolnie długą rozgrywkę nawet przy mocy zbioru $A$ równej 9.


\section{Komputerowa implementacja Online Thue Game}

Komputerowa implementacja gry Thuego opiera się na pomyśle gry z pracy [8]. W grze dostępnych jest kilka trybów zarówno dla jednego oraz dwóch graczy jak i komputerowa symulacja, czyli gra komputera przeciwko niemu samemu.

Implementacja gry Online Thue Game nazywana będzie \textbf{Longest Square-Free word}.
Zasady pozostają niemal identyczne. Na początku gry, gracze ustalają moc zbioru symboli, oraz otrzymują swoje role. Jeden z nich staje się architektem, drugi malarzem. Rola architekta polega na wybieraniu odpowiedniego indeksu w ciągu tworzonym przez graczy, pod którym powstanie nowy element. Malarz natomiast określa symbol wstawianego elementu. Gra kończy się w momencie, gdy w ciągu tworzonym przez graczy pojawia się \textbf{kwadrat}. Indeks $i$  podawany przez architekta nie może być mniejszy od zera oraz większy niż $n$, gdzie $n$ jest równe liczbie elementów w ciągu. Na początku gry ciąg $S$ zawiera tylko jeden symbol równy 0.
Grę rozpoczyna architekt. Gracze wykonują swoje ruchy na przemian. Malarz otrzymuje punkt za każdy pomalowany element, który nie tworzy \textbf{kwadratu} wewnątrz ciągu.
By wyłonić zwycięzcę potrzebne są dwie rundy. Każdy z graczy musi sprawdzić się zarówno jako malarz i architekt. Wygrywa osoba, która zdobyła więcej punktów jako malarz.

Tryb ten dostępny jest również dla jednego gracza, rolę przeciwnika otrzymuje wtedy komputer, który działa według algorytmu przewidującego określoną przez poziom trudności liczbę ruchów do przodu. Algorytm może pełnić zarówno rolę budowniczego jak i malarza.

Bliźniaczym trybem gry, opierającym się na tych samych zasadach z niewielką różnicą jest \textbf{Longest Overlap-Free word}. 
Różnica polega na tym, że malarz w tworzonym ciągu musi unikać \textbf{nasunięcia}.

Podobnie jak w \textbf{Longest Square-Free word} jest możliwość gry przeciwko algorytmowi, który jest w stanie przewidywać określoną ilość ruchów do przodu.

\chapter{Aplikacja Longest free word}
Program, który powstał na bazie gier Longest Square-free word oraz Longest Overlap-free word, został napisany w Javie. Umożliwia on rozgrywkę w obu wersjach gry, zarówno z drugim graczem, jak i z komputerem, oraz jest w stanie zasymulować rozgrywkę dwóch graczy komputerowych, grających przeciwko sobie z ustawionymi przez nas poziomami inteligencji. Dodatkową opcją jest uruchomienie obszernego testu, który zasymuluje rozgrywkę komputerowych graczy na wszystkich poziomach trudności, mniejszych od sprecyzowanego przez nas $n$. Komunikacja z aplikacją odbywa się poprzez wspomniany plik konfiguracyjny - przed uruchomieniem aplikacji, oraz konsolę aplikacji - po jej uruchomieniu. W programie zostało użyte narzędzie automatyzujące budowę aplikacji - Maven. Dzięki niemu, po pobraniu kodu źródłowego aplikacji jesteśmy w stanie uruchomić ją z poziomu konsoli dzięki dwóm linijkom wpisanym do terminala. Okazało się ono niezwykle pomocne podczas przeprowadzania czasochłonnych testów na maszynie zdalnej, której sterowanie odbywało się właśnie poprzez terminal.


\section{Plik konfiguracyjny}
Opcje dostępne wewnątrz pliku konfiguracyjnego to:
\begin{itemize}
\item \textbf{gameType} - wartości, jakie możemy wprowadzić to Square oraz Overlap. Jest to typ gry, którego zamierzamy użyć i precyzuje, czy zagramy w Longest Square-free word czy Longest Overlap-free word.
\item \textbf{gameMode} - tryb gry wartości wpisywane w tym polu mają wpływ na to, czy gra odbywać się będzie z drugim człowiekiem - \textit{humanHuman}, komputerem z tym warunkiem, że to my jesteśmy budowniczym - \textit{humanBuilder}, ponownie z komputerem, jednak tym razem to on jest budowniczym - \textit{pcBuilder}, oraz walka dwóch komputerów - \textit{pcPc}.
\item \textbf{setPower} - jest to moc zbioru elementów, do których dostęp będzie miał malarz podczas rozgrywki. Zbiór ten wypełniany jest liczbami $\in \{0, ...,n-1\}$
\item \textbf{builderNestingLevel} i \textbf{painterNestingLevel} - poziom zagnieżdżenia na jaki komputer sobie pozwoli jako budowniczy i malarz. Zasada działania zagnieżdżeń zostanie omówiona w dalszej części pracy.
\item \textbf{maxThinkTime} - podczas przeprowadzania testów z udziałem komputerowych graczy, czasami nie chcemy by przeciwnik myślał nad swoim ruchem 17 godzin. Właśnie dlatego została wprowadzona ta opcja konfiguracyjna. Jeśli czas jaki komputer spędził nad wyliczeniem kolejnej pozycji lub symbolu, będzie większy niż ustalona przez nas liczba nanosekund rozgrywka zostaje przerwana.
\item \textbf{makeOverallTest} - gdy opcja ta zostanie ustawiona na true, po uruchomieniu aplikacji zostanie przeprowadzony obszerny test, zawierający w sobie kombinacje rozgrywek komputerów o wszystkich poziomach zagnieżdżeń $\leq 6$ na wszystkich mocach zbiorów $>0 $ i $\leq 7$. Czyli zostanie wykonanych $6*6*6$ rozgrywek. Warto wspomnieć, że podczas tego testu opcja maxThinkTime okazała się niezwykle pomocna.
\end{itemize}

\section{Algorytm szukający powtórzeń wewnątrz ciągu}


\definecolor{dkgreen}{rgb}{0,0.6,0}
\definecolor{gray}{rgb}{0.5,0.5,0.5}
\definecolor{mauve}{rgb}{0.58,0,0.82}


\lstset{frame=tb,
  language=Java,
  aboveskip=3mm,
  belowskip=3mm,
  showstringspaces=false,
  columns=flexible,
  basicstyle={\small\ttfamily},
  numbers=left,
  numberstyle=\tiny\color{black},
  keywordstyle=\color{blue},
  commentstyle=\color{dkgreen},
  stringstyle=\color{mauve},
  breaklines=true,
  breakatwhitespace=true,
  tabsize=3,
}

\lstset{
    inputencoding=utf8x, 
    extendedchars=\true,
    literate=%
    {ą}{{\k{a}}}1
    {Ą}{{\k{A}}}1
    {ę}{{\k{e}}}1
    {Ę}{{\k{E}}}1
    {ó}{{\'o}}1
    {Ó}{{\'O}}1
    {ś}{{\'s}}1
    {Ś}{{\'S}}1
    {ł}{{\l{}}}1
    {Ł}{{\L{}}}1
    {ż}{{\.z}}1
    {Ż}{{\.Z}}1
    {ź}{{\'z}}1
    {Ź}{{\'Z}}1
    {ć}{{\'c}}1
    {Ć}{{\'C}}1
    {ń}{{\'n}}1
    {Ń}{{\'N}}1
}

\begin{lstlisting}[frame=single]
List<Integer> findSquare(List<Integer> sequence) {
	List<Integer> squareSeq = null;
	int maxSeqSize = sequence.size()/2;
	int minSeqSize = 1;
	for(int subSeąSize=minSeqSize; subSeqSize<=maxSeqSize;subSeqSize++) {
		squareSeq = compareSubSeq(subSeqSize, sequence);
		if(squareSeq != null) {
			return squareSeq;
		}
	}
	return null;
}
\end{lstlisting}

Powyższy algorytm ma za zadanie znalezienie kwadratu w sekwencji, którą reprezentuje lista Integerów przekazana w parametrze. Zmienna \textbf{maxSeqSize} jest to długość najdłuższego podciągu jaki się zmieści w sekwencji, jeśli dostawimy za nim podciąg o identycznej długości.

Natomiast zmienna \textbf{minSeqSize} jest to minimalna długość podciągu, który może składać się na kwadrat, naturalnie wynosi ona 1. 

W linii 5 wykonujemy pętlę, wewnątrz, której do metody \textbf{compareSubSeq} przekazywana jest długość podciągu, który składać się będzie na kwadrat, oraz naszą sekwencję. Metoda \textbf{compareSubSeq} zwróci nam lewą część znalezionego kwadratu, lub null w przypadku, gdy taki kwadrat w sekwencji nie istnieje.

\begin{lstlisting}[frame=single]
Subsequence compareSubSeq(int subSeqSize, List<Integer> sequence) {
	List<Integer> left = new ArrayList<>();
	List<Integer> right = new ArrayList<>();
	int comparesFitInSequence = (sequence.size() + 1) - (subSeqSize*2) ;
	for(int i=0; i<comparesFitInSequence; i++) {
		for(int j =0;j<subSeqSize;j++) {
			left.add(sequence.get(i+j));
			right.add(sequence.get(i+j+subSeqSize));
		}
		if(listsAreEqual(left, right)) {
			return new Subsequence(left, i, subSeqSize);
		}
		left.clear();
		right.clear();
	}
	return null;
}
\end{lstlisting}
Zmienne \textbf{left} i \textbf{right} są to podciągi, które reprezentują lewą i prawą część kwadratu $aa$. Zmienna \textbf{comparesFitInSequence} jest to ilość porównań jaka zmieści się wewnątrz naszej sekwencji. Dla przykładu, gdy nasz ciąg, jest reprezentowany przez $S = abcabaca$, a wcześniej sprecyzowana długość podciągu, składającego się na kwadrat wynosi 2, to zmienna \textbf{comparesFitInSequence} zostanie ustawiona na $(8 + 1) - (2 * 2)$ czyli 5, ponieważ możliwe są następujące porównania: $\textbf{abca}baca$, $a\textbf{bcab}aca$, $ab\textbf{caba}ca$, $abc\textbf{abac}a$, $abca\textbf{baca}$
W pętli znajdującej się w linii 5 do listy \textbf{left} oraz \textbf{right} dodawane są podciągi odpowiedniej długości, natomiast w linii 10 następuje sprawdzenie czy podciągi są identyczne. Jeśli - tak oznacza to, że w naszej sekwencji, rozpoczynając od indeksu $i$ występuje kwadrat długości $2 * \textbf{subSeqSize}$. Jeśli listy są różne to w linii 13 następuje ich wyczyszczenie, po to by pętla mogła porównać dwa kolejne podciągi.

Działanie algorytmu ilustruje grafika:

\begin{center}
\includegraphics[scale = 0.2]{images/squareFinding}
\end{center}

Dla typu gry w którym unikamy nasunięć powstały osobne metody.

\begin{lstlisting}[frame=single]
Subsequence findOverlap(List<Integer> sequence) {
	Subsequence repeatedSequence = null;
	int maxSeqSize = (sequence.size()/2)+1;
	int minSeqSize = 3;
	for(int subSeqSize=minSeqSize; subSeqSize<=maxSeqSize; subSeqSize++) {
		repeatedSequence = compareSubSeqOverlap(subSeqSize, sequence);
		if(repeatedSequence != null) {
			return repeatedSequence;
		}
	}
	return null;
}
\end{lstlisting}

Metoda \textbf{findOverlap}, działa w sposób podobny do metody \textbf{findSquare}. Różnice to inne wartości zmiennych \textbf{maxSeqSize}, \textbf{minSeqSize} oraz metoda wywoływana w pętli.

Do zmiennej \textbf{maxSeqSize} przypisywana jest liczba o jeden większa niż długość sekwencji, ponieważ szukamy nasunięcia, a więc podciągi będą ze sobą dzieliły jeden znak. Wobec tego dla ciągu $S = \{abcabcb\}$, najdłuższy porównywany ciąg będzie długości 4, ponieważ w ostatniej iteracji pętli będziemy ze sobą porównywali podciągi $abca$ oraz $abcb$, które dzielą ze sobą literę $a$.

Wartość zmiennej \textbf{minSeqSize} wynosi 3, ponieważ jest to warunkiem stworzenia nasunięcia.

\begin{lstlisting}[frame=single]

Subsequence compareSubSeqOverlap(int subSeqSize, List<Integer> sequence) {
	List<Integer> left = new ArrayList<>();
	List<Integer> right = new ArrayList<>();
	int comparesFitInSequence = (sequence.size() + 2) - (subSeqSize*2);
	for(int i=0; i<comparesFitInSequence; i++) {
		for(int j =0;j<subSeqSize;j++) {
			left.add(sequence.get(i+j));
			right.add(sequence.get(i+j+subSeqSize-1));
		}
		if(listsAreEqual(left, right)) {
			return new Subsequence(left, i, subSeqSize);
		}
		left.clear();
		right.clear();
	}
	return null;
} 
\end{lstlisting}

Metoda \textbf{compareSubSeqOverlap} również jest analogiczna do metody \textbf{compareSubSeq}. Różni się tutaj wartość zmiennej \textbf{comparesFitInSequence}, jest ona większa o 1, z takiego samego powodu, co zmienna \textbf{maxSeqSize} z metody \textbf{findOverlap}. Różni się również podciąg zapisywany do zmiennej \textbf{right}, w pętli z 6 linii. Pierwszy indeks owego podciągu jest równy indeksowi ostatniego elementu, lewego podciągu, po to by stworzyć nasunięcie.

Działanie algorytmu ilustruje grafika:
\begin{center}
\includegraphics[scale = 0.2]{images/overlapFinding}
\end{center}

\section{Komunikacja z użytkownikiem}
Plik konfiguracyjny nie jest wystarczającym środkiem komunikacji z aplikacjią. W związku z tym, po uruchomieniu programu mamy dostęp do konsoli, służy do wyświetlania jak i wprowadzania treści.

Po uruchomieniu aplikacji wypisywane w konsoli wypisywane są najważniejsze opcje konfiguracyjne, takie jak poziom budowniczego, poziom malarza oraz lista dostępnych symboli. Jeśli uruchomiliśmy grę w trybie \textbf{humanHuman}, to aplikacja w pierwszej kolejności poprosi nas o indeks. Po wpisaniu indeksu mieszczącego się w przedziale $\in \{0,...,n\}$, gdy to zrobimy zostaniemy poproszeni o podanie symbolu $\geq 0$ i $< setPower$. W momencie gdy podaliśmy prawidłowe wartości na ekran konsoli wyświetlany zostaje ciąg $S'$, który został utworzony poprzez dodanie do istniejącego ciągu odpowiedniego symbolu. Jeśli w ciągu $S'$ pojawi się kwadrat, na ekran zostanie wyświetlony czas rozgrywki, ilość ruchów jaka została do tej pory wykonana oraz indeksy wraz z symbolami, wspomnianego powtórzenia.

Oto przykład prostej rozgrywki:

\begin{lstlisting}
#> W grze dostępne są następujące liczby: 
0
1
2
 0: { 0 }  1: {   } 
## =============================================================== ##
#> Podaj indeks: 
1
#> Podaj liczbę: 
2
 0: { 0 }  1: { 2 }  2: {   } 
## =============================================================== ##
#> Podaj indeks: 
1
#> Podaj liczbę: 
1
 0: { 0 }  1: { 1 }  2: { 2 }  3: {   } 
## =============================================================== ##
#> Podaj indeks: 
3
#> Podaj liczbę: 
1
 0: { 0 }  1: { 1 }  2: { 2 }  3: { 1 }  4: {   } 
## =============================================================== ##
#> Podaj indeks: 
4
#> Podaj liczbę: 
2
 0: { 0 }  1: { 1 }  2: { 2 }  3: { 1 }  4: { 2 }  5: {   } 
## =============================================================== ##

#> Znaleziono kwadrat:  
 1: { 1 }  2: { 2 }  <->  3: { 1 }  4: { 2 } 

## =============================================================== ##

#> Rozrgrywka trwała 5 ruchów i 20.339 sekund.
\end{lstlisting}
Jeśli uruchomimy rozgrywkę z komputerem, obok wybranego indeksu lub symbolu pojawia się również czas jaki był mu potrzebny na podjęcie decyzji.

\begin{lstlisting}
#> Komputer wybrał indeks: 1 	| Czas trwania obliczeń: 24.127 s
#> Komputer wybrał liczbę: 4 	| Czas trwania obliczeń: 0.012 s\end{lstlisting}

By ułatwić późniejszą analizę wszystkie informacje zawarte w konsoli, zapisywane są do nowo utworzonego pliku w katalogu output.

\section{Zachłanny algorytm wyszukiwania symbolu}
Jak zostało wcześniej wspomniane można sterować poziomem inteligencji komputerowych oponentów za pomocą zmiennych konfiguracyjnych. Jeśli ustawimy poziom zagnieżdżeń na 0, algorytm będzie działał zachłannie, wybierając opcje, która jest najatrakcyjniejsza w danym momencie, nie zważając na to, co może wydarzyć się w kolejnej turze. Obrazuje to poniższy algorytm szukający pasującego symbolu na wskazanym wcześniej indeksie.

\begin{lstlisting}[frame=single]
int findRightColorGreedy(List<Integer> sequence, int index, int power) {
		for(int symbol=0;symbol<power; symbol++) {
			sequence.add(index, symbol);
			if(pickProperFind(sequence) == null) {
				sequence.remove(index);
				return symbol;
			} else {
				sequence.remove(index);
			}
		}
		return -1;
	}
\end{lstlisting}

Metoda na wejściu dostaje trzy parametry:
\begin{itemize}
\item sequence - utworzony wcześniej ciąg.
\item index - indeks wybrany przez budowniczoego.
\item power - moc zbioru symboli.
\end{itemize}
Zbiór symboli to liczby należące do przedziału $[0; power-1]$. W 2 linii metody rozpoczyna się pętla, która iteruje po wszystkich dostępnych symbolach. W kolejnej linii symbol dodawany jest do naszego ciągu na podanej pozycji. Metoda pickProperFind z warunku if zależnie od typu gry wywołuje wcześniej opisane metody findSquare lub findOverlap. Jeśli warunek if zostanie spełniony oznacza to, że po dodaniu aktualnego symbolu na danej pozycji nie powoduje stworzenia kwadratu/nasunięcia, algorytm zatem usuwa dodany element z ciągu i zwraca go w linii 6. Jeżeli  okaże się jednak, że dodany symbol tworzy powtórzenie w ciągu, zostaje on również usunięty, a pętla zaczyna się od początku. Metoda zwraca wartość -1 jeśli okaże się, że żaden z symboli nie jest w stanie stworzyć ciągu wolnego od kwadratów/nasunięć.

\section{Zachłanny algorytm wyszukiwania indeksu}
By można było przeprowadzić symulację gry, należy wprowadzić również algorytm budowczniego, starający się znaleźć najmniej atrakcyjny indeks dla malarza. Zachłanny algorytm realizujący to zadanie znajduje się poniżej.

\begin{lstlisting}[frame=single]
int findRightIndexGreedy(List<Integer> sequence, int power) {
	int winner = -1;
	int smallestSymbolSize = power;
	for (int i =0;i<sequence.size()+1;i++) {
		List<Integer> symbols = getFitableColorList(sequence, i);
		if (symbols.size() <= smallestSymbolSize) {
			smallestSymbolSize = symbols.size();
			winner = i;
		}
	}
	return winner;
}
\end{lstlisting}

Algorytm jako parametry, otrzymuje stworzony wcześniej ciąg, oraz moc zbioru symboli. Zmienna winner ustawiona początkowo na -1, reprezentuje indeks, który zostanie zwrócony jako ten, pod którym jest najmniejsza dowolność wyboru symboli, bez tworzenia powtórzeń. Metoda iteruje po każdym indeksie ciągu i zapisuje do listy symbole, po których wstawieniu w dane miejsce nie utworzy się kwadrat/nasunięcie. Następnie w warunku if z 6 linii sprawdzane jest, czy aktualna lista symboli jest mniejsza niż ta, zarejestrowana wcześniej. Jeśli tak, do zmiennej winner zapisany zostaje aktualny indeks.

\section{Algorytm wyszukiwania symbolu z zagnieżdżeniami}
Algorytm zachłanny nie jest wystarczająco sprytny, żeby przeciwstawić się człowiekowi mającemu odrobinę doświadczenia w Longest free word. Wobec tego powstał algorytm nie działający zachłannie, lecz starający się przewidzieć, jakie konsekwencje w kolejnych turach może nieść ze sobą dany wybór.

Poniższy algorytm wprowadza pojęcie punktacji. Podczas jego działania, dla możliwych wyborów nadawane są punkty. Im więcej punktów uzyska dany symbol, tym atrakcyjniejszym staje się on wyborem. Algorytm sprawdza jakie symbole można dodać w predefiniowanym indeksie, następnie iterując w pętli, dodaje każdy z symboli i sprawdza, ile możliwości będzie miał w kolejnej lub kolejnych turach, biorąc pod uwagę wszystkie dostępne indeksy. Każda dodatkowa możliwość to dodatkowy punkt dla wybranego koloru. Sam algorytm składa się z dwóch głownych części: \textbf{findRightColorPredicting} oraz \textbf{simulation}.

\begin{lstlisting}[frame=single]
int findRightColorPredicting(List<Integer> sequence, int index) {
	List<Integer> scoreList = initScoreList(power);
	List<Integer> symbolList = getFitableSymbolList(sequence, index);
	for(int symbol: symbolList) {
		sequence.add(index, symbol);
		simulation(sequence, symbol, scoreList, painterNestingLevel);
		sequence.remove(index);
	}
	return getRandomFromScoreList(scoreList);
}
\end{lstlisting}

Na początku algorytm inicjalizuje listy \textbf{scoreList} oraz \textbf{symbolList}. Tę pierwszą tyloma zerami ile jest w grze dostępnych symboli, drugą natomiast symbolami, jakie możemy wstawić w zdefiniowany przez budowniczego indeks. Linia 4 rozpoczyna pętlę, która iterując po liście \textbf{symbolList}, dodaje jej element, wywołuje metodę \textbf{simulation}, przekazując utworzony \textbf{ciąg}, dodany \textbf{symbol}, \textbf{scoreList} oraz p\textbf{oziom zagnieżdżenia malarza}, po czym usuwa dodany \textbf{symbol} sprawiając, że ciąg pozostaje bez zmian. Na końcu zwraca element, który miał największą liczbę punktów. Jeśli kilka symboli otrzymało ich tyle samo, program wybiera losowy z nich. Dzięki w grze występuje pewna przypadkowość, oraz jest mała szansa na powtórzenie dwóch identycznych rozgrywek przy odpowiednio długim ciągu.

Wewnątrz metody \textbf{simulation} nadawane są punkty, oraz za pomocą rekurencji wykonywana jest symulacja kolejnych iteracji gry.

\begin{lstlisting}[frame=single]
void simulation(List<Integer> sequence, int indexInScoreList, List<Integer> scoreList, int invokes) {
	invokes--;
	for(int j=0;j<sequence.size()+1;j++) {
		List<Integer> symbolList = getFitableSymbolList(sequence, j);
		updateScoreList(scoreList, indexInScoreList, symbolList.size());
		for(int symbol: symbolList) {
			sequence.add(j, symbol);
			if(invokes > 0) {
				simulation(sequence, indexInScoreList, scoreList, invokes);
			}
			sequence.remove(j);
		}
	}
}
\end{lstlisting}

Na początku dekrementowana zostaje ilość zagnieżdżeń na jaką chcemy się zagłębić. Oznacza to, że jeśli poziom zagnieżdżenia ustawiony jest na 1, to metoda \textbf{simulation} zostanie wywołana tylko raz. Pętla z 3 linii iteruje po indeksach pod którymi możliwe jest dodanie symbolu. W 4 linii do listy zapisywane są symbole, które można wcisnąć pod aktualny indeks, nie powodując powtórzenia. Metoda \textbf{updateScoreList} zwiększa ilość punktów symbolu przekazanemu z poprzedniej metody. Ilość punktów jest równa liczbie elementów, zmiennej \textbf{symbolList}. Pętla z 6 linii iterując po liście pasujących symboli, dodaje element, następnie pod warunkiem, że \textbf{invokes} jest większe od 0 wywołuje samą siebie ze zmodyfikowanym ciągiem, tym samym indeksem, który został przekazany na początku, listą punktową oraz pozostałą liczbą wywołań. Na końcu pętli element zostaje usunięty, po to, by ciąg wrócił do pierwotnego stanu.

Algorytm obrazuje poniższa grafika:

\begin{center}
\includegraphics[scale = 0.08]{images/nestingPainter}
\end{center}

Mamy ciąg $S = a, b$, architekt wybrał indeks 2, czyli malarz wybiera symbol jaki zostanie wstawiony za literką $b$. Typ gry to \textbf{Longest square free word}. Moc zbioru wynosi 3, więc do wyboru są dostępne symbole $a$, $b$, $c$. Wstawienie symbolu $b$ stworzyłoby kwadrat, dlatego algorytm rozważa literki $a$ oraz $c$. Zmienna $n$ jest to poziom zagnieżdżenia algorytmu, natomiast liczby przy pasujących symbolach oznaczają punkty, jakie zostają przypisane symbolowi $a$ lub $c$. 

Już przy pierwszym zagnieżdżeniu widać, że literka $c$ jest atrakcyjniejsza, ponieważ wstawienie jej zapewnia malarzowi więcej możliwości w kolejnych rundach. Jednak sprawdzenie jednego ruchu do przodu nie zawsze jest wystarczające i zdarza się, że z pozoru nieatrakcyjny symbol w perspektywie kolejnych, tur jest najlepszą opcją.

\section{Algorytm wyszukiwania indeksu z zagnieżdżeniami}
Sposób działania algorytmu z zagnieżdżeniami szukającego indeksu, który ma największą szansę na stworzenie powtórzenia, działa na podobnej zasadzie, co algorytm wyszukiwania symbolu. Różnica polega na tym, że w jego pierwszej części, iterujemy po wszystkich dostępnych symbolach, a nie jednym predefiniowanym, oraz punkty nadawane zostają indeksom i zwracamy ten, który uzyska ich najmniej. Druga część algorytmu czyli metoda \textbf{simulation} pozostaje bez zmian.

\begin{lstlisting}[frame=single]
int findRightIndex(List<Integer> sequence) {
	List<Integer> scoreList = initScoreList(sequence.size() + 1);
	for (int i = 0; i<sequence.size() + 1; i++) {
		List<Integer> symbols = getFitableSymbolList(sequence, i);
		for (int symbol : symbols) {
			sequence.add(i, symbol);
			simulation(sequence, i, predictList, builderNestingLevel);
			sequence.remove(i);
		}
	}
	return getRandomMinFromPredict(scoreList);
}
\end{lstlisting}

Na początku metody zainicjalizowana zostaje zmienna scoreList. Do listy dodane zostają zera w ilość odpowiadającej długość ciągu plus jeden, bo właśnie na tylu indeksach możemy dodać nowy element. W 3 linii iterujemy po każdym dostępnym indeksie, natomiast od 4 linii mamy już wszystko to co w algorytmie szukającym symbolu. Na końcu metody zwracany zostaje indeks, który uzyskał najmniej punktów, czyli będzie najmniej atrakcyjny dla malarza. Tak jak poprzednio jeśli istnieje więcej niż jeden indeks z minimalna wartością, element jest wybierany losowo.

\section{Longest square free word z interfejsem graficznym}
Powstała również wersja gry, w której użytkownik nie korzysta z konsoli, lecz z graficznego interfejsu. W tej odmianie użytkownik za pomocą trzech kolorów ma stworzyć jak najdłuższy ciąg bez kwadratów. Czynności jakie gracz wykonuje to wybranie miejsca, w które zostanie wstawiony element oraz jego kolor. Jeśli w tworzonym ciągu pojawi się kwadrat rozgrywka zostaje przerwana i na ekranie zostają podświetlone powtórzenia składające się na kwadrat. Program zapisuje również najwyższy wynik, który jest równy długości utworzonego ciągu. Algorytm szukający kwadratów nie różni się w żaden sposób od algorytmu z podstawowej wersji gry.

\begin{center}
\includegraphics[scale = 0.2]{images/thueMobile}
\end{center}

Aplikacja została napisana na popularnym silniku do tworzenia gier - Unity. Pozwala on kompilować kod programu do plików wykonywalnych, które mogą być uruchamiane w przeglądarkach, na komputerach osobistych, konsolach i telefonach komórkowych. W tym przypadku pod uwagę brane były głównie telefony komórkowe, o czym może świadczyć wielkość przycisków i mała ilość szczegółów na ekranie.


\chapter{Analiza symulowanych potyczek} 

\section{Ilość symboli potrzebna na rozgranie partii}
Długość gry zależna jest od dwóch czynników. Ustawionego poziomu zagnieżdżenia i ilości dostępnych symboli. Thue w swoim twierdzeniu[][], udowodnił, że jesteśmy w stanie stworzyć nieskończenie długi ciąg bez kwadratów mając do dyspozycji 3 symbole. Sytuacja zmienia się, gdy ciąg tworzony jest przez dwóch oponentów, z których jeden stara się utworzyć kwadrat. Nawet używając zachłannego algorytmu w przypadku budowniczego i algorytmu z zagnieżdżeniami 7 poziomu dla malarza, rozgrywka zawsze trwała będzie maksymalnie 5 ruchów. 

Dla 4 symboli, również nie jest możliwe rozegranie partii dłuższej niż 10 ruchów. Przy zachłannym algorytmie budowniczego, malarz jest w stanie utworzyć ciąg długości 10. Jeśli algorytm budowniczego używa zagnieżdżeń, liczba ta zmniejsza się do 7.

Dopiero rozgrywka na 5 symbolach daje algorytmom pole do popisu, bowiem w przeprowadzonych badaniach, zależnie od ustawionych poziomów, trwała ona od 11 do conajmniej 236 ruchów.

\begin{table}[!h]
\begin{tabular}{|l|l|l|l|} \hline
Ilość symboli & Malarz & Budowniczy & Ilość ruchów \\ \hline
$3$ & $\geq 0$ & $\geq 0$ & $5$\\ \hline
$4$ & $0$ & $0$ & $7$\\ \hline
$4$ & $\geq 1 $ & $\geq 1$ & $7$\\ \hline
$4$ & $0$ & $\geq 1$ & $10$\\ \hline
\end{tabular}
\end{table}

\hfill \break
Używanie zachłannego algorytmu malarza nadal nie skutkuje zbyt długą grą. Niezależnie od poziomu budowniczego, kwadrat pojawi się zawsze po 11 kroku. Jednak podczas prób, gdy malarz miał ustawiony poziom zagnieżdżenia równy 1, a budowniczy działał zachłannie, rozgrywka trwała od 44 do 236 tur. Średnia arytmetyczna wyciągnięta z ilości ruchów wyniosła $133.83$

\begin{table}[!h]
\begin{tabular}{|l|l|l|} \hline
Ilość symboli & Ilość ruchów & Czas trwania \\ \hline
5 & 44 & 0.1 s.\\ \hline
5 & 56 & 0.4 s.\\ \hline
5 & 81 & 2.4 s.\\ \hline
5 & 88 & 4.3 s.\\ \hline
5 & 100 & 7.6 s.\\ \hline
5 & 102 & 7.3 s.\\ \hline
5 & 163 & 1 min. 27 s. \\ \hline
5 & 171 & 1 min. 48 s. \\ \hline
5 & 173 & 1 min. 59 s. \\ \hline
5 & 193 & 4 min. \\ \hline
5 & 199 & 3 min. 38 s.\\ \hline
5 & 236 & 8 min. 51 s. \\ \hline
\end{tabular}
\end{table}

\hfill \break
\hfill \break
\hfill \break
\hfill \break
\hfill \break
\hfill \break
\hfill \break
\hfill \break
\hfill \break
\hfill \break
\hfill \break
\hfill \break
Przy poziomie budowniczego i malarza ustawionym na 1, podczas 10 prób rozgrywka trwała od 30 do 213 ruchów. Średnia arytmetyczna obliczona na podstawie rozgrywek wynosi $96.9$, a więc jest o $37.23$ niższa niż, gdy budowniczy używał algorytmu zachłannego. 

\begin{table}[!h]
\begin{tabular}{|l|l|l|} \hline
Ilość symboli & Ilość ruchów & Czas trwania \\ \hline
5 & 30 & 1.2 s. \\ \hline
5 & 42 & 5.1 s. \\ \hline
5 & 45 & 11 s. \\ \hline
5 & 47 & 7.8 s. \\ \hline
5 & 73 & 1 min. 17 s. \\ \hline
5 & 104 & 9 min. 38 s. \\ \hline
5 & 114 & 29 min. 29 s. \\ \hline
5 & 134 & 1 godz. 30 min. 27 s.  \\ \hline
5 & 178 & 5 godz. 11 min. 22 s. \\ \hline
5 & 202 & 9 godz. 42 min. 15 s. \\ \hline
5 & 213 & 13 godz. 04 min. 35 s. \\ \hline
\end{tabular}
\end{table}


Obecne zasoby sprzętowe i czasowe sprawiły, że malarz poziomu 2 okazał się niepokonany w walce z budowniczym poziomu 0. Przy ponad 10 próbach ani razu nie był zmuszony do stworzenia kwadratu w rozgrywce trwającej $\leq 220$ tur i około 26 godzin.





\iffalse
Budowniczy 0
Malarz 1
Dla 3 kolorów 
#> Rozrgrywka trwała 5 ruchów i 0.006 sekund.

Budowniczy 0
Malarz 1
Dla 4 kolorów
#> Rozrgrywka trwała 10 ruchów i 0.01 sekund.

Budowniczy 0
Malarz 1
Dla 5 kolorów
#> Rozrgrywka trwała 44 ruchów i 0.188 sekund.
#> Rozrgrywka trwała 56 ruchów i 0.478 sekund.
#> Rozrgrywka trwała 81 ruchów i 2.423 sekund.
#> Rozrgrywka trwała 88 ruchów i 4.325 sekund.
#> Rozrgrywka trwała 100 ruchów i 7.636 sekund.
#> Rozrgrywka trwała 102 ruchów i 7.313 sekund.
#> Rozrgrywka trwała 163 ruchów i 87.994 sekund.
#> Rozrgrywka trwała 171 ruchów i 108.328 sekund.
#> Rozrgrywka trwała 173 ruchów i 119.859 sekund.
#> Rozrgrywka trwała 193 ruchów i 240.487 sekund.
#> Rozrgrywka trwała 199 ruchów i 218.112 sekund.
#> Rozrgrywka trwała 236 ruchów i 531.952 sekund.


Budowniczy 0
Malarz 2
#> Przekroczono limit czasu oczekiwania na decyzje komputera: 1845.38 s
#> Rozrgrywka trwała 208 ruchów i 39545.025 sekund.

#> Przekroczono limit czasu oczekiwania na decyzje komputera: 1826.289 s
#> Rozrgrywka trwała 212 ruchów i 48841.542 sekund.

#> Przekroczono limit czasu oczekiwania na decyzje komputera: 595.425 s
#> Rozrgrywka trwała 166 ruchów i 12562.789 sekund.

#> Przekroczono limit czasu oczekiwania na decyzje komputera: 607.311 s
#> Rozrgrywka trwała 170 ruchów i 13488.263 sekund.


#> Przekroczono limit czasu oczekiwania na decyzje komputera: 9485.516 s
#> Rozrgrywka trwała 222 ruchów i 100280.632 sekund.

#> Przekroczono limit czasu oczekiwania na decyzje komputera: 7305.829 s
#> Rozrgrywka trwała 222 ruchów i 100617.55 sekund.

#> Przekroczono limit czasu oczekiwania na decyzje komputera: 11991.49 s
#> Rozrgrywka trwała 223 ruchów i 103534.735 sekund.

#> Przekroczono limit czasu oczekiwania na decyzje komputera: 7209.71 s
#> Rozrgrywka trwała 225 ruchów i 107490.89 sekund.

#> Przekroczono limit czasu oczekiwania na decyzje komputera: 7223.661 s
#> Rozrgrywka trwała 225 ruchów i 107208.699 sekund.


Budowniczy 1
Malarz 0

Dla 4 kolorów
#> Rozrgrywka trwała 7 ruchów i 0.017 sekund.
7 a nie 10


Dla 5 kolorów
Budowniczy 0
Malarz 0
#> Rozrgrywka trwała 11 ruchów i 0.015 sekund.

Budowniczy 1
Malarz 0

#> Rozrgrywka trwała 11 ruchów i 0.045 sekund.


Budowniczy 1
Malarz 1
#> Rozrgrywka trwała 30 ruchów i 1.238 sekund.
#> Rozrgrywka trwała 42 ruchów i 5.1 sekund.
#> Rozrgrywka trwała 45 ruchów i 11.982 sekund.
#> Rozrgrywka trwała 47 ruchów i 7.898 sekund.
#> Rozrgrywka trwała 73 ruchów i 77.12 sekund.
#> Rozrgrywka trwała 104 ruchów i 578.538 sekund.
#> Rozrgrywka trwała 114 ruchów i 1769.184 sekund.
#> Rozrgrywka trwała 134 ruchów i 5427.881 sekund.
#> Rozrgrywka trwała 178 ruchów i 18682.459 sekund.
#> Rozrgrywka trwała 202 ruchów i 34935.922 sekund.
#> Rozrgrywka trwała 213 ruchów i 47075.211 sekund.


budowniczy 2
malarz 1

#> Rozrgrywka trwała 24 ruchów i 15.267 sekund.
#> Rozrgrywka trwała 63 ruchów i 7997.669 sekund.
#> Rozrgrywka trwała 83 ruchów i 36676.259 sekund.




\fi

\section{Czas oczekiwania na decyzje}


\section{Obserwacja zachowań algorytmu budowniczego}

\chapter{Narzędzia i~standardy pokrewne}

Systemy SGML, ze względu na mnogość funkcji jakie spełniają i~ich
kompleksowe podejście do oznakowywania i~przetwarzania dokumentów
tekstowych, są bardzo skomplikowane. Możemy wyróżnić dwa podejścia do
budowy takich systemów.  Z~jednej strony, buduje się systemy
zindywidualizowane, oparte o~specyficzne narzędzia tworzone w~takich
językach, jak: C, C++, Perl czy Python. Edytory strukturalne, filtry
do transformacji formatów czy parsery\index{parser} i~biblioteki
przydatne do konstrukcji dalszych narzędzi, tworzone są według potrzeb
określonych, pojedynczych systemów.

Z~drugiej strony, twórcy oprogramowania postanowili pójść krok dalej
i~połączyć te różne narzędzia w~jedną całość. Tą całość miał stanowić
DSSSL lub jego XML-owy odpowiednik -- standard XSL. Ze względu na
oferowane możliwości można twierdzić, że tworzenie i~używanie narzędzi
implementujących standard DSSSL/XSL, jest najwłaściwszym
podejściem. Przemawiają za tym różne argumenty, ale najważniejszym
z~nich jest to, że mamy tu możliwość stworzenia niezależnego od
platformy programowej i~narzędziowej zbioru szablonów -- przepisów jak
przetwarzać dokumenty SGML.

\section{Przetwarzanie dokumentów SGML -- standard DSSSL\label{s:dsssl}}

DSSSL (\textit{Document Style Semantics and Specification Language\/})
-- to międzynarodowy standard ściśle związany ze standardem SGML.
Standard ten, można podzielić na następujące części:

\begin{itemize}
\item język transformacji (\textit{transformation language\/}).  To
  definicja języka służącego do transformacji dokumentu oznaczonego
  znacznikami zgodnie z~pewnym DTD na dokument oznaczony zgodnie
  z~innym~DTD.
\item język stylu (\textit{style language\/}) opisujący sposób
  formatowania dokumentów SGML.
\item język zapytań (\textit{query language\/}) służy do
  identyfikowania poszczególnych fragmentów dokumentu SGML.
\end{itemize}

Opisane główne części składowe standardu DSSSL dają obraz tego, jak
wiele aspektów przetwarzania zostało zdefiniowanych i~jak
skomplikowany jest to problem. Jest to głównym powodem tego, że mimo
upływu kilku lat od zdefiniowania standardu nie powstały ani
komercyjne ani wolnodostępne aplikacje wspierające go
w~całości. Istnieją natomiast \emph{nieliczne\/} narzędzia realizujące
DSSSL w~ograniczonym zakresie, głównie w~części definiującej język
stylu, który odpowiada za opatrzenie dokumentu czysto strukturalnego
w~informacje formatujące. Daje to możliwość publikacji dokumentów SGML
zarówno w~postaci elektronicznej, hipertekstowej czy też drukowanej.

\section{Przetwarzanie dokumentów XML -- standard XSL\label{s:xsl}}

Tak jak XML jest \emph{uproszczoną\/} wersją standardu SGML, tak XSL
jest uproszczonym odpowiednikiem standardu DSSSL. W~szczególności,
wyróżnić można w~tym standardzie następujące części składowe:

\begin{itemize}
\item język transformacji (XSLT) To definicja języka służącego do
  transformacji dokumentu.
\item język zapytań (XPath) służy do identyfikowania poszczególnych
  fragmentów dokumentu.
\item język stylu deefiniujący sposób formatowania dokumentów XML.
\end{itemize}

\chapter{Przegląd dostępnych narzędzi\label{PRZEGLAD.NARZEDZI}}

W~celu wykorzystania standardu SGML do przetwarzania dokumentów,
niezbędne jest zebranie odpowiedniego zestawu narzędzi. Narzędzi do
przetwarzania dokumentów SGML jest wiele. Są to zarówno całe
systemy zintegrowane, jak i~poszczególne programy, biblioteki czy
skrypty wspomagające.

\section{Narzędzia do przeglądania dokumentów SGML}

Do tej kategorii oprogramowania zaliczamy przeglądarki dokumentów
SGML oraz serwery sieciowe wspomagające standard SGML, przy
czym rozwiązań wspierających standard XML jest już w~chwili obecnej
dużo więcej i~są dużo powszechniejsze.

Jeżeli chodzi o~przeglądarki to zarówno Internet Explorer jak
i~Netscape umożliwiają bezpośrednie wyświetlenie dokumentów XML;
ponieważ jednak nie wspierają w~całości standardu XML, prowadzi to
ciągle do wielu problemów\footnote{Z~innych mniej popularnych
  rozwiązań można wymienić takie aplikacje, jak: HyBrick SGML
  Browser firmy Fujitsu Limited, Panorama Publisher firmy InterLeaf
  Inc, DynaText firmy Inso Corporation czy darmowy QWeb. W~przypadku
  serwerów zwykle dokonują one transformacji ,,w~locie'' żądanych
  dokumentów na format HTML (rzadziej bezpośrednio wyświetlają
  dokumenty XML).  Ta kategoria oprogramowania ma, z~punktu widzenia
  projektu, znaczenie drugorzędne.}.

\section{Parsery SGML}
Program \texttt{nsgmls} (z~pakietu \texttt{SP} Jamesa Clarka) jest
doskonałym parserem\index{parser} dokumentów SGML, dostępnym
publicznie.  Parser \texttt{nsgmls} jest dostępny w~postaci źródłowej
oraz w~postaci programów wykonywalnych przygotowanych na platformę
MS~Windows, Linux/Unix i~inne. Oprócz analizy poprawności dokumentu
parser\index{parser} ten umożliwia również konwersję danych do formatu
ESIS\index{ESIS}, który wykorzystywany jest jako dane wejściowe przez
wiele narzędzi do przetwarzania i~formatowania dokumentów SGML.
Dodatkowymi, bardzo przydatnymi elementami pakietu \texttt{SP} są:
program \texttt{sgmlnorm} do normalizacji, program \texttt{sx} służący
do konwersji dokumentu SGML na XML oraz biblioteki programistyczne,
przydatne przy tworzeniu specjalistycznych aplikacji służących do
przetwarzania dokumentów SGML.

W~przypadku dokumentów XML publicznie dostępnych, parserów jest
w~chwili obecnej kilkadziesiąt. Do popularniejszych zaliczyć można
Microsoft Java XML Parser firmy Microsoft, LT XML firmy Language
Technology Group, Exapt oraz XP (James Clark)

\section{Wykorzystanie języków skryptowych}

\section{Wykorzystanie szablonów XSL}

Stosując wersję XML typu DocBook można wykorzystać szablony stylów
przygotowane w~standardzie XSL (autor N.~Walsh). W~chwili obecnej
są dostępne narzędzia umożliwiające przetworzenie dokumentów XML do
postaci drukowanej (Adobe PDF) oraz hipertekstowej (HTML).

Podobnie jak w~przypadku szablonów DSSSL, szablony stylów XSL są
sparametryzowane i~udokumentowane i~dzięki temu łatwe w~adaptacji. Do
zamiany dokumentu XML na postać prezentacyjną można wykorzystać jeden
z~dostępnych publicznie procesorów XSLT
(por.~tabela~\ref{zest:proces:xslt}).

\begin{table}[!htb]
\begin{tabular}{|l|l|l|} \hline
Nazwa & Autor      & Adres URL \\ \hline
\texttt{sablotron} & Ginger Alliance & \url{http://www.gingerall.com} \\ \hline
\texttt{Xt}        & J.~Clark & \url{http://www.jclark.com} \\ \hline
\texttt{4XSLT}     & FourThought & \url{http://www.fourthought.com} \\ \hline
\texttt{Saxon}     & Michael Kay &  \url{http://users.iclway.co.uk/mhkay/saxon} \\ \hline
\texttt{Xalan}     & Apache XML Project & \url{http://xml.apache.org} \\ \hline
\end{tabular}
\caption{Publicznie dostępne procesory XLST\label{zest:proces:xslt}}
\source{Opracowanie własne}
\end{table}

XSL:FO jest skomplikowanym językiem o~dużych możliwościach,
zawierającym ponad 50 różnych ,,obiektów formatujących'', począwszy od
najprostszych, takich jak prostokątne bloki tekstu poprzez wyliczenia,
tabele i~odsyłacze. Obiekty te można formatować wykorzystując przeszło
200 różnych właściwości (\emph{properties\/}), takich jak: kroje,
odmiany i~wielkości pisma, odstępy, kolory itp.
W~tym dokumencie przedstawione jest absolutne miniumum informacji
na temat standardu XSL:FO.

Cały dokument XSL:FO zawarty jest wewnątrz elementu \texttt{fo:root}.
Element ten zawiera (w~podanej niżej kolejności):

\begin{itemize}
\item dokładnie jeden element \texttt{fo:layout-master-set} zawierający
  szablony określające wygląd poszczególnych stron oraz sekwencji
  stron (te ostatnie są opcjonalne, ale typowo są definiowane);
\item zero lub więcej elementów \texttt{fo:declarations};
\item jeden lub więcej elementów \texttt{fo:page-sequance}
 zawierających treść formatowanego dokumentu wraz z~opisem
 jego sformatowania i~podziału na strony.
\end{itemize}

% zakończenie
\summary
Możliwości, jakie stoją przed archiwum prac magisterskich opartych na
XML-u, są ograniczone jedynie czasem, jaki należy poświęcić na pełną
implementację systemu. Nie ma przeszkód technologicznych do stworzenia
co najmniej równie doskonałego repozytorium, jak ma to miejsce w
przypadku ETD. Jeżeli chcemy w pełni uczestniczyć w rozwoju nowej ery
informacji, musimy szczególną uwagę przykładać do odpowiedniej
klasyfikacji i archiwizacji danych. Sądzę, że język XML znacznie to
upraszcza.

% załączniki (opcjonalnie):
\appendix
\chapter{Tytuł załącznika jeden}

Treść załącznika jeden.

\chapter{Tytuł załącznika dwa}

Treść załącznika dwa.

% literatura (obowiązkowo):
\bibliographystyle{unsrt}
\bibliography{xml}

% spis tabel (jeżeli jest potrzebny):
\listoftables

% spis rysunków (jeżeli jest potrzebny):
\listoffigures

\oswiadczenie

\end{document}
